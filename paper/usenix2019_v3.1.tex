%%%%%%%%%%%%%%%%%%%%%%%%%%%%%%%%%%%%%%%%%%%%%%%%%%%%%%%%%%%%%%%%%%%%%%%%%%%%%%%%
% Template for USENIX papers.
%
% History:
%
% - TEMPLATE for Usenix papers, specifically to meet requirements of
%   USENIX '05. originally a template for producing IEEE-format
%   articles using LaTeX. written by Matthew Ward, CS Department,
%   Worcester Polytechnic Institute. adapted by David Beazley for his
%   excellent SWIG paper in Proceedings, Tcl 96. turned into a
%   smartass generic template by De Clarke, with thanks to both the
%   above pioneers. Use at your own risk. Complaints to /dev/null.
%   Make it two column with no page numbering, default is 10 point.
%
% - Munged by Fred Douglis <douglis@research.att.com> 10/97 to
%   separate the .sty file from the LaTeX source template, so that
%   people can more easily include the .sty file into an existing
%   document. Also changed to more closely follow the style guidelines
%   as represented by the Word sample file.
%
% - Note that since 2010, USENIX does not require endnotes. If you
%   want foot of page notes, don't include the endnotes package in the
%   usepackage command, below.
% - This version uses the latex2e styles, not the very ancient 2.09
%   stuff.
%
% - Updated July 2018: Text block size changed from 6.5" to 7"
%
% - Updated Dec 2018 for ATC'19:
%
%   * Revised text to pass HotCRP's auto-formatting check, with
%     hotcrp.settings.submission_form.body_font_size=10pt, and
%     hotcrp.settings.submission_form.line_height=12pt
%
%   * Switched from \endnote-s to \footnote-s to match Usenix's policy.
%
%   * \section* => \begin{abstract} ... \end{abstract}
%
%   * Make template self-contained in terms of bibtex entires, to allow
%     this file to be compiled. (And changing refs style to 'plain'.)
%
%   * Make template self-contained in terms of figures, to
%     allow this file to be compiled. 
%
%   * Added packages for hyperref, embedding fonts, and improving
%     appearance.
%   
%   * Removed outdated text.
%
%%%%%%%%%%%%%%%%%%%%%%%%%%%%%%%%%%%%%%%%%%%%%%%%%%%%%%%%%%%%%%%%%%%%%%%%%%%%%%%%

\documentclass[letterpaper,twocolumn,10pt]{article}
\usepackage{usenix2019_v3}

% to be able to draw some self-contained figs
\usepackage{tikz}
\usepackage{amsmath}

% inlined bib file
\usepackage{filecontents}

%-------------------------------------------------------------------------------
% \begin{filecontents}{\jobname.bib}
%-------------------------------------------------------------------------------
%@Book{arpachiDusseau18:osbook,
%  author =       {Arpaci-Dusseau, Remzi H. and Arpaci-Dusseau Andrea C.},
%  title =        {Operating Systems: Three Easy Pieces},
%  publisher =    {Arpaci-Dusseau Books, LLC},
%  year =         2015,
%  edition =      {1.00},
%  note =         {\url{http://pages.cs.wisc.edu/~remzi/OSTEP/}}
%}
%@InProceedings{waldspurger02,
%  author =       {Waldspurger, Carl A.},
%  title =        {Memory resource management in {VMware ESX} server},
%  booktitle =    {USENIX Symposium on Operating System Design and
%                  Implementation (OSDI)},
%  year =         2002,
%  pages =        {181--194},
%  note =         {\url{https://www.usenix.org/legacy/event/osdi02/tech/waldspurger/waldspurger.pdf}}}
% \end{filecontents}

%-------------------------------------------------------------------------------
\begin{document}
%-------------------------------------------------------------------------------

%don't want date printed
\date{}

\title{\Large Adversarial Neural Networks in Traffic Obfuscation}

\author{
{\rm Steven R. Sheffey}\\
Middle Tennessee State University
\and
{\rm Ferrol Aderholdt}\\
Middle Tennessee State University
} % end author

\maketitle

%-------------------------------------------------------------------------------
\begin{abstract}
%-------------------------------------------------------------------------------
% Your abstract text goes here. Just a few facts. Whet our appetites.
% Not more than 200 words, if possible, and preferably closer to 150.
We live in a society. Gamers rise up! Israel will no longer oppress us.
\end{abstract}


%-------------------------------------------------------------------------------
\section{Introduction}
%-------------------------------------------------------------------------------

Internet censorship is bad

The TOR network helps circumvent internet censorship

Internet censors block tor

Pluggable transports, such as Meek hide tor traffic as benign-looking HTTPS traffic

However, this obfuscated traffic is easily identifiable through machine learning attacks on traffic patterns

%-------------------------------------------------------------------------------
\section{Background}
%-------------------------------------------------------------------------------

Censors block tor

tor introduced pluggable transports to disguise tor traffic to make it harder to block

Meek, one of these transports hides TOR traffic inside the encrypted payload of an HTTPS connection, using a technique known as domain fronting

Wang et al differentiated Meek traffic from HTTPs traffic by using machine learning algorithms on traffic pattern statistics

Obfuscating Meek traffic in a way that is resilient to these techniques remains an open problem


%-------------------------------------------------------------------------------
\section{Related Work}
foo
%-------------------------------------------------------------------------------
\section{Methods}
%-------------------------------------------------------------------------------

foo
%-------------------------------------------------------------------------------
\section{Results}
%-------------------------------------------------------------------------------
foo

%-------------------------------------------------------------------------------
\section{Conclusions}
%-------------------------------------------------------------------------------
foo
%-------------------------------------------------------------------------------
\section{Availability}
%-------------------------------------------------------------------------------

The traffic generation framework, feature extractor, and machine learning code for this work is open source, and can be accessed at \texttt{https://github.com/starfys/packet\_captor\_sakura}
%-------------------------------------------------------------------------------
% \bibliographystyle{plain}
% \bibliography{\jobname}

%%%%%%%%%%%%%%%%%%%%%%%%%%%%%%%%%%%%%%%%%%%%%%%%%%%%%%%%%%%%%%%%%%%%%%%%%%%%%%%%
\end{document}
%%%%%%%%%%%%%%%%%%%%%%%%%%%%%%%%%%%%%%%%%%%%%%%%%%%%%%%%%%%%%%%%%%%%%%%%%%%%%%%%

%%  LocalWords:  endnotes includegraphics fread ptr nobj noindent
%%  LocalWords:  pdflatex acks
